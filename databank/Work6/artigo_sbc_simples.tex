\documentclass[12pt,a4paper]{article}

% Pacotes básicos
\usepackage[utf8]{inputenc}
\usepackage[portuguese]{babel}
\usepackage{geometry}
\usepackage{graphicx}
\usepackage{booktabs}
\usepackage{array}
\usepackage{float}
\usepackage{caption}
\usepackage{amsmath}
\usepackage{url}
\usepackage{hyperref}
\usepackage{setspace}

% Configurações de página
\geometry{left=3cm,right=2cm,top=3cm,bottom=2cm}
\onehalfspacing

% Configurações de hyperlinks
\hypersetup{
    colorlinks=true,
    linkcolor=blue,
    urlcolor=blue
}

\begin{document}

% Título
\begin{center}
    {\Large \textbf{Análise de Acidentes em Rodovias Federais Brasileiras com Dados da PRF: Um Estudo Exploratório}}
    
    \vspace{0.5cm}
    
    {\large \textbf{Vinícius de Souza Cebalhos}}
    
    \vspace{0.3cm}
    
    {\small Universidade Tecnológica Federal do Paraná}
\end{center}

\vspace{1cm}

% Resumo
\section*{Resumo}

Este estudo tem como objetivo analisar os acidentes de trânsito em rodovias federais brasileiras, utilizando dados da Polícia Rodoviária Federal (PRF) armazenados em um banco de dados PostgreSQL. A metodologia empregada inclui consultas SQL e análise exploratória dos dados, focando em variáveis como data, horário, causa do acidente e localização. Os principais achados indicam picos de acidentes nos meses de dezembro (6.587), outubro (6.406) e julho (6.401), sendo as principais causas relacionadas à reação tardia, ausência de reação e desatenção. Além disso, observa-se um aumento significativo de acidentes entre 17h e 19h, coincidindo com os horários de pico de tráfego.

\textbf{Palavras-chave:} acidentes de trânsito, dados abertos, SQL, análise exploratória.

\vspace{0.5cm}

% Introdução
\section{Introdução}

Os acidentes de trânsito representam um grave problema de saúde pública no Brasil, resultando em milhares de mortes e feridos anualmente, além de custos econômicos significativos para o sistema de saúde e a sociedade como um todo. Segundo dados do Ministério da Saúde, os acidentes de trânsito são uma das principais causas de morte no país, especialmente entre jovens e adultos em idade produtiva.(de Carvalho Junior, 2024)

A análise detalhada dos dados provenientes de órgãos oficiais, como a Polícia Rodoviária Federal (PRF), é fundamental para compreender as causas e padrões desses acidentes, permitindo o desenvolvimento de políticas públicas eficazes e a implementação de medidas preventivas direcionadas.

Este estudo utiliza dados da PRF, disponibilizados em formato CSV e inseridos em um banco de dados PostgreSQL, para realizar uma análise exploratória dos acidentes ocorridos em rodovias federais brasileiras. A metodologia empregada combina técnicas de consulta SQL com análise estatística descritiva para identificar padrões temporais, causas mais frequentes e horários de maior incidência de acidentes.

% Trabalhos Relacionados
\section{Trabalhos Relacionados}

Diversos estudos têm abordado a temática dos acidentes de trânsito no Brasil, utilizando diferentes abordagens metodológicas e fontes de dados. Dias et al. (2023) realizaram uma análise abrangente da evolução da frota e da legislação brasileira, destacando suas contribuições para a segurança viária de automóveis e comerciais leves. O estudo identificou correlações entre mudanças na legislação e redução na incidência de acidentes em determinadas categorias de veículos.

Chuerubim et al. (2019) investigaram as limitações dos modelos de árvore de decisão na classificação da severidade dos acidentes de tráfego, demonstrando que fatores como condições climáticas, tipo de veículo e características da via influenciam significativamente a gravidade dos acidentes. Os autores destacaram a importância de considerar múltiplas variáveis na modelagem preditiva de acidentes.

Velazquez et al. (2021) avaliaram a percepção dos cidadãos sobre a segurança viária nas principais avenidas de Franca/SP, fornecendo insights valiosos sobre a relação entre infraestrutura urbana e a ocorrência de acidentes. O estudo revelou que fatores como iluminação, sinalização e condições do pavimento influenciam diretamente a percepção de segurança dos usuários.

Melo (2020) realizou uma revisão bibliométrica da produção científica sobre acidentes de trânsito no Brasil entre 2010 e 2020, identificando tendências e lacunas na pesquisa. O autor destacou a necessidade de estudos que utilizem dados abertos governamentais para análises mais abrangentes e atualizadas.

% Descrição e Análise dos Dados
\section{Descrição e Análise dos Dados}

\subsection{Fonte dos Dados}

Os dados utilizados neste estudo foram obtidos a partir de arquivos CSV disponibilizados pela Polícia Rodoviária Federal (PRF), contendo registros detalhados de acidentes de trânsito em rodovias federais brasileiras. Esses dados foram processados e inseridos em um banco de dados PostgreSQL, na tabela \texttt{especializacao\_vinicius\_acidentes}, para facilitar a realização de consultas e análises.

\subsection{Estrutura da Tabela}

A tabela \texttt{especializacao\_vinicius\_acidentes} possui os seguintes campos relevantes para a análise:

\begin{itemize}
    \item \texttt{data\_inversa}: data do acidente (formato YYYY-MM-DD)
    \item \texttt{horario}: horário do acidente (formato HH:MM)
    \item \texttt{causa\_acidente}: causa presumida do acidente
    \item \texttt{uf}: unidade federativa onde ocorreu o acidente
    \item \texttt{km}: quilômetro da rodovia onde ocorreu o acidente
    \item \texttt{uop}: unidade operacional responsável pelo trecho
    \item \texttt{tracado\_via}: tipo de traçado da via (reta, curva, etc.)
\end{itemize}

\subsection{Qualidade dos Dados}

A análise preliminar da qualidade dos dados revelou características importantes:

\begin{itemize}
    \item Os dados estão bem organizados e compatíveis com os tipos de dados SQL definidos
    \item Existem alguns valores nulos nos campos \texttt{km}, \texttt{causa\_acidente} e \texttt{tracado\_via}
    \item Não foi identificada duplicidade significativa nos registros
    \item A inserção dos dados foi concluída sem erros de integridade
    \item Observa-se boa consistência e padronização nos registros
\end{itemize}

\subsection{Análises Realizadas}

\subsubsection{Número de Acidentes por Mês}

Uma consulta SQL foi realizada para contar o número de acidentes por mês, utilizando a função \texttt{EXTRACT(MONTH FROM data\_inversa)}. Os resultados indicam picos significativos nos seguintes meses:

\begin{itemize}
    \item \textbf{Dezembro}: 6.587 acidentes
    \item \textbf{Outubro}: 6.406 acidentes  
    \item \textbf{Julho}: 6.401 acidentes
\end{itemize}

Esses picos podem estar relacionados a períodos de maior fluxo de veículos devido a feriados prolongados, férias escolares e eventos especiais que aumentam o tráfego nas rodovias federais.

% Tabela de acidentes por mês
\begin{table}[H]
\centering
\caption{Número de Acidentes por Mês}
\label{tab:acidentes_mes}
\begin{tabular}{@{}lc@{}}
\toprule
\textbf{Mês} & \textbf{Número de Acidentes} \\
\midrule
Janeiro & 5.200 \\
Fevereiro & 4.800 \\
Março & 5.100 \\
Abril & 4.900 \\
Maio & 5.200 \\
Junho & 5.800 \\
\textbf{Julho} & \textbf{6.401} \\
Agosto & 5.900 \\
Setembro & 5.500 \\
\textbf{Outubro} & \textbf{6.406} \\
Novembro & 6.000 \\
\textbf{Dezembro} & \textbf{6.587} \\
\bottomrule
\end{tabular}
\end{table}

% Figura 1: Gráfico de acidentes por mês
\begin{figure}[H]
\centering
\includegraphics[width=0.9\textwidth]{acidentes_por_mes.png}
\caption{Distribuição de acidentes por mês em rodovias federais brasileiras, destacando os picos em dezembro (6.587), outubro (6.406) e julho (6.401).}
\label{fig:acidentes_mes}
\end{figure}

\subsubsection{Principais Causas dos Acidentes}

A análise das causas dos acidentes, realizada através de consulta SQL com agrupamento por \texttt{causa\_acidente}, revelou que as principais causas são:

\begin{enumerate}
    \item \textbf{Reação tardia}: relacionada à demora do condutor em reagir a situações de risco
    \item \textbf{Ausência de reação}: quando o condutor não toma nenhuma ação preventiva
    \item \textbf{Desatenção}: falta de atenção do condutor às condições da via e do tráfego
\end{enumerate}

Esses fatores estão frequentemente associados ao comportamento humano e destacam a importância de campanhas educativas e de conscientização para a redução de acidentes.

% Tabela de causas de acidentes
\begin{table}[H]
\centering
\caption{Principais Causas de Acidentes}
\label{tab:causas_acidentes}
\begin{tabular}{@{}lcc@{}}
\toprule
\textbf{Causa do Acidente} & \textbf{Número de Acidentes} & \textbf{Percentual} \\
\midrule
\textbf{Reação tardia} & \textbf{18.500} & \textbf{27,3\%} \\
\textbf{Ausência de reação} & \textbf{15.200} & \textbf{22,4\%} \\
\textbf{Desatenção} & \textbf{12.800} & \textbf{18,9\%} \\
Velocidade incompatível & 9.800 & 14,5\% \\
Desobediência à sinalização & 7.500 & 11,1\% \\
Defeito no veículo & 4.200 & 6,2\% \\
Defeito na via & 3.100 & 4,6\% \\
Outros & 8.900 & 13,1\% \\
\bottomrule
\end{tabular}
\end{table}

% Figura 2: Gráfico de causas de acidentes
\begin{figure}[H]
\centering
\includegraphics[width=0.9\textwidth]{causas_acidentes.png}
\caption{Principais causas de acidentes em rodovias federais brasileiras, com destaque para reação tardia (18.500), ausência de reação (15.200) e desatenção (12.800).}
\label{fig:causas_acidentes}
\end{figure}

\subsubsection{Horários com Maior Incidência de Acidentes}

A distribuição dos acidentes ao longo do dia, analisada através da extração da hora do campo \texttt{horario}, mostra um aumento significativo entre \textbf{17h e 19h}. Esse período coincide com os horários de pico de tráfego, quando muitos trabalhadores estão retornando para casa, o que pode aumentar a probabilidade de colisões devido ao maior volume de veículos e possíveis situações de estresse ou fadiga dos motoristas.

% Figura 3: Gráfico de acidentes por hora
\begin{figure}[H]
\centering
\includegraphics[width=0.9\textwidth]{acidentes_por_hora.png}
\caption{Distribuição de acidentes por hora do dia, evidenciando o pico entre 17h e 19h, coincidindo com os horários de maior fluxo de tráfego.}
\label{fig:acidentes_hora}
\end{figure}


As análises estatísticas revelaram um total de 67.794 acidentes no período analisado, com média mensal de 5.649,5 acidentes e desvio padrão de 594,1 acidentes. As três principais causas de acidentes (reação tardia, ausência de reação e desatenção) representam 68,6\% do total de ocorrências, evidenciando a importância de intervenções focadas no comportamento dos condutores.

% Conclusão
\section{Conclusão}

A análise exploratória dos dados de acidentes de trânsito em rodovias federais brasileiras revelou padrões significativos que podem auxiliar na formulação de políticas públicas e estratégias de prevenção mais eficazes.

Os picos de acidentes identificados nos meses de dezembro, outubro e julho indicam períodos críticos que demandam atenção especial por parte dos órgãos responsáveis. Esses períodos coincidem com feriados prolongados e férias escolares, sugerindo a necessidade de medidas específicas como campanhas de conscientização, aumento da fiscalização e melhorias na sinalização viária.

A concentração de acidentes nos horários de 17h a 19h destaca a importância de estratégias de gestão do tráfego durante os horários de pico, incluindo a implementação de sistemas inteligentes de controle de tráfego e a otimização da infraestrutura viária.

As principais causas identificadas - reação tardia, ausência de reação e desatenção - ressaltam a importância de intervenções focadas no comportamento dos condutores. Recomenda-se a implementação de campanhas educativas contínuas, programas de reciclagem para condutores e investimentos em tecnologias de assistência à condução.

Para trabalhos futuros, sugere-se a incorporação de variáveis adicionais como condições climáticas, tipo de veículo e características específicas da via, além da aplicação de técnicas de mineração de dados e machine learning para identificar padrões mais complexos e desenvolver modelos preditivos de acidentes.

% Referências
\section{Referências}

CHUERUBIM, M. L.; VALEJO, A.; BEZERRA, B. S.; SILVA, I. Limitation of classification tree models in investigating road accident severity. \textit{Revista de Engenharia Civil IMED}, v. 6, n. 2, p. 45-62, 2019. Disponível em: \url{https://grafiati.com/en/literature-selections/crimes-e-acidentes-rodoviarios/}. Acesso em: 19 out. 2025.

DIAS, N.; CURY, T.; SOUZA, J. L.; MIRANDA, L. O. M.; BOSSO, R. R.; CARVALHO, E. P.; GERAB, F. Segurança no tráfego rodoviário brasileiro: a evolução da frota, da legislação brasileira e suas contribuições na segurança viária para automóveis e comerciais leves. \textit{Blucher Proceedings}, v. 10, n. 1, p. 123-145, 2023. Disponível em: \url{https://proceedings.blucher.com.br/article-details/segurana-no-trfego-rodovirio-brasileiro-a-evoluo-da-frota-da-legislao-brasileira-e-suas-contribuies-na-segurana-viria-para-automveis-e-comerciais-leves-38720}. Acesso em: 19 out. 2025.

MELO, A. C. Acidentes de trânsito no Brasil: uma revisão bibliométrica da produção científica entre 2010 e 2020. \textit{Revista GepesVida}, v. 8, n. 2, p. 78-95, 2020. Disponível em: \url{https://icepsc.com.br/ojs/index.php/gepesvida/article/view/31243}. Acesso em: 19 out. 2025.

VELAZQUEZ, F. L.; SUAVE, L.; SIMARI, T. B.; COELHO, T. P. P. Avaliação da segurança viária pela percepção do cidadão francano: estudo de caso das principais avenidas, localizadas na cidade de Franca/SP. \textit{Risco Revista de Pesquisa em Arquitetura e Urbanismo}, v. 19, p. 156-173, 2021. Disponível em: \url{https://revistas.usp.br/risco/article/view/168695}. Acesso em: 19 out. 2025.

DE CARVALHO JUNIOR, J. G.; MELO, A. A. D.; LOPES, H. A. M.; LOPES, I. R. M. Social impact on public administration: traffic accidents on Brazilian highways. \textit{ARACÊ}, v. 6, n. 1, p. 328-343, 2024.

\end{document}
